\documentclass[journal,12pt,twocolumn]{IEEEtran}
%
\usepackage{setspace}
\usepackage{gensymb}
\usepackage{xcolor}
\usepackage{caption}
%\usepackage{subcaption}
%\doublespacing
\singlespacing

%\usepackage{graphicx}
%\usepackage{amssymb}
%\usepackage{relsize}
\usepackage[cmex10]{amsmath}
\usepackage{mathtools}
%\usepackage{amsthm}
%\interdisplaylinepenalty=2500
%\savesymbol{iint}
%\usepackage{txfonts}
%\restoresymbol{TXF}{iint}
%\usepackage{wasysym}
\usepackage{amsthm}
\usepackage{mathrsfs}
\usepackage{txfonts}
\usepackage{stfloats}
\usepackage{cite}
\usepackage{cases}
\usepackage{subfig}
%\usepackage{xtab}
\usepackage{longtable}
\usepackage{multirow}
%\usepackage{algorithm}
%\usepackage{algpseudocode}
\usepackage{enumitem}
\usepackage{mathtools}
%\usepackage{iithtlc}
%\usepackage[framemethod=tikz]{mdframed}
\usepackage{listings}


%\usepackage{stmaryrd}


%\usepackage{wasysym}
%\newcounter{MYtempeqncnt}
\DeclareMathOperator*{\Res}{Res}
%\renewcommand{\baselinestretch}{2}
\renewcommand\thesection{\arabic{section}}
\renewcommand\thesubsection{\thesection.\arabic{subsection}}
\renewcommand\thesubsubsection{\thesubsection.\arabic{subsubsection}}

\renewcommand\thesectiondis{\arabic{section}}
\renewcommand\thesubsectiondis{\thesectiondis.\arabic{subsection}}
\renewcommand\thesubsubsectiondis{\thesubsectiondis.\arabic{subsubsection}}

% correct bad hyphenation here
\hyphenation{op-tical net-works semi-conduc-tor}

\lstset{
language=Python,
frame=single, 
breaklines=true
}

%\lstset{
	%%basicstyle=\small\ttfamily\bfseries,
	%%numberstyle=\small\ttfamily,
	%language=python,
	%backgroundcolor=\color{white},
	%%frame=single,
	%%keywordstyle=\bfseries,
	%%breaklines=true,
	%%showstringspaces=false,
	%%xleftmargin=-10mm,
	%%aboveskip=-1mm,
	%%belowskip=0mm
%}

%\surroundwithmdframed[width=\columnwidth]{lstlisting
\begin{document}
%

\theoremstyle{definition}
\newtheorem{theorem}{Theorem}[section]
\newtheorem{problem}{Problem}[section]
\newtheorem{proposition}{Proposition}[section]
\newtheorem{lemma}{Lemma}[section]
\newtheorem{corollary}[theorem]{Corollary}
\newtheorem{example}{Example}[section]
\newtheorem{definition}{Definition}[section]
%\newtheorem{definition}{Definition}
%\newtheorem{algorithm}{Algorithm}[section]
%\newtheorem{cor}{Corollary}
\newcommand{\BEQA}{\begin{eqnarray}}
\newcommand{\EEQA}{\end{eqnarray}}
\newcommand{\define}{\stackrel{\triangle}{=}}

\bibliographystyle{IEEEtran}
%\bibliographystyle{ieeetr}

\providecommand{\nCr}[2]{\,^{#1}C_{#2}} % nCr
\providecommand{\nPr}[2]{\,^{#1}P_{#2}} % nPr
\providecommand{\mbf}{\mathbf}
\providecommand{\pr}[1]{\ensuremath{\Pr\left(#1\right)}}
\providecommand{\qfunc}[1]{\ensuremath{Q\left(#1\right)}}
\providecommand{\sbrak}[1]{\ensuremath{{}\left[#1\right]}}
\providecommand{\lsbrak}[1]{\ensuremath{{}\left[#1\right.}}
\providecommand{\rsbrak}[1]{\ensuremath{{}\left.#1\right]}}
\providecommand{\brak}[1]{\ensuremath{\left(#1\right)}}
\providecommand{\lbrak}[1]{\ensuremath{\left(#1\right.}}
\providecommand{\rbrak}[1]{\ensuremath{\left.#1\right)}}
\providecommand{\cbrak}[1]{\ensuremath{\left\{#1\right\}}}
\providecommand{\lcbrak}[1]{\ensuremath{\left\{#1\right.}}
\providecommand{\rcbrak}[1]{\ensuremath{\left.#1\right\}}}
\theoremstyle{remark}
\newtheorem{rem}{Remark}
\newcommand{\sgn}{\mathop{\mathrm{sgn}}}
\providecommand{\abs}[1]{\left\vert#1\right\vert}
\providecommand{\res}[1]{\Res\displaylimits_{#1}} 
\providecommand{\norm}[1]{\lVert#1\rVert}
\providecommand{\mtx}[1]{\mathbf{#1}}
\providecommand{\mean}[1]{E\left[ #1 \right]}
\providecommand{\fourier}{\overset{\mathcal{F}}{ \rightleftharpoons}}
%\providecommand{\hilbert}{\overset{\mathcal{H}}{ \rightleftharpoons}}
\providecommand{\system}{\overset{\mathcal{H}}{ \longleftrightarrow}}
	%\newcommand{\solution}[2]{\textbf{Solution:}{#1}}
\newcommand{\solution}{\noindent \textbf{Solution: }}
\providecommand{\dec}[2]{\ensuremath{\overset{#1}{\underset{#2}{\gtrless}}}}
\numberwithin{equation}{section}
%\numberwithin{equation}{problem}
%\numberwithin{problem}{subsection}
%\numberwithin{definition}{subsection}
\providecommand{\mean}[1]{E\left[ #1 \right]}
\providecommand{\fourier}{\overset{\mathcal{F}}{ \rightleftharpoons}}
%\providecommand{\hilbert}{\overset{\mathcal{H}}{ \rightleftharpoons}}
\providecommand{\system}{\overset{\mathcal{H}}{ \longleftrightarrow}}
	%\newcommand{\solution}[2]{\textbf{Solution:}{#1}}
%\newcommand{\solution}{\noindent \textbf{Solution: }}
\providecommand{\dec}[2]{\ensuremath{\overset{#1}{\underset{#2}{\gtrless}}}}
\numberwithin{equation}{section}
%\numberwithin{equation}{problem}
%\numberwithin{problem}{subsection}
%\numberwithin{definition}{subsection}
\makeatletter
\@addtoreset{figure}{problem}
\makeatother

\let\StandardTheFigure\thefigure
%\renewcommand{\thefigure}{\theproblem.\arabic{figure}}
\renewcommand{\thefigure}{\theproblem}


%\numberwithin{figure}{subsection}

\def\putbox#1#2#3{\makebox[0in][l]{\makebox[#1][l]{}\raisebox{\baselineskip}[0in][0in]{\raisebox{#2}[0in][0in]{#3}}}}
     \def\rightbox#1{\makebox[0in][r]{#1}}
     \def\centbox#1{\makebox[0in]{#1}}
     \def\topbox#1{\raisebox{-\baselineskip}[0in][0in]{#1}}
     \def\midbox#1{\raisebox{-0.5\baselineskip}[0in][0in]{#1}}

\vspace{3cm}

\title{
%\logo{
EE5603:Conceention Inequalities
%}
%	\logo{python for Math Computing }
}
%\title{
%	\logo{Matrix Analysis through python}{\begin{center}\includegraphics[scale=.24]{tlc}\end{center}}{}{HAMDSP}
%}


% paper title
% can use linebreaks \\ within to get better formatting as desired
%\title{Matrix Analysis through python}
%
%
% author names and IEEE memberships
% note positions of commas and nonbreaking spaces ( ~ ) LaTeX will not break
% a structure at a ~ so this keeps an author's name from being broken across
% two lines.
% use \thanks{} to gain access to the first footnote area
% a separate \thanks must be used for each paragraph as LaTeX2e's \thanks
% was not built to handle multiple paragraphs
%

\author{J.~Balasubramaniam$^{\dagger}$ and G V V Sharma$^{*}$ %<-this  stops a space
\thanks{$\dagger$ The author is with the Department of Mathematics, IIT Hyderabad.  *The author is with the Department
of Electrical Engineering, IIT, Hyderabad
502285 India e-mail: \{jbala,gadepall\}@iith.ac.in. All material in the manuscript is released under GNU GPL.  Free to use for all.}% <-this % stops a space
%\thanks{J. Doe and J. Doe are with Anonymous University.}% <-this % stops a space
%\thanks{Manuscript received April 19, 2005; revised January 11, 2007.}}
}
% note the % following the last \IEEEmembership and also \thanks - 
% these prevent an unwanted space from occurring between the last author name
% and the end of the author line. i.e., if you had this:
% 
% \author{....lastname \thanks{...} \thanks{...} }
%                     ^------------^------------^----Do not want these spaces!
%
% a space would be appended to the last name and could cause every name on that
% line to be shifted left slightly. This is one of those "LaTeX things". For
% instance, "\textbf{A} \textbf{B}" will typeset as "A B" not "AB". To get
% "AB" then you have to do: "\textbf{A}\textbf{B}"
% \thanks is no different in this regard, so shield the last } of each \thanks
% that ends a line with a % and do not let a space in before the next \thanks.
% Spaces after \IEEEmembership other than the last one are OK (and needed) as
% you are supposed to have spaces between the names. For what it is worth,
% this is a minor point as most people would not even notice if the said evil
% space somehow managed to creep in.



% The paper headers
%\markboth{Journal of \LaTeX\ Class Files,~Vol.~6, No.~1, January~2007}%
%{Shell \MakeLowercase{\textit{et al.}}: Bare Demo of IEEEtran.cls for Journals}
% The only time the second header will appear is for the odd numbered pages
% after the title page when using the twoside option.
% 
% *** Note that you probably will NOT want to include the author's ***
% *** name in the headers of peer review papers.                   ***
% You can use \ifCLASSOPTIONpeerreview for conditional compilation here if
% you desire.
% If you want to put a publisher's ID mark on the page you can do it like
% this:
%\IEEEpubid{0000--0000/00\$00.00~\copyright~2007 IEEE}
% Remember, if you use this you must call \IEEEpubidadjcol in the second
% column for its text to clear the IEEEpubid mark.



% make the title area
\maketitle
\section{Convergence}
\subsection{Definitions}
%\begin{itemize}
%\item Today :\item review
%\complete proof of benntl's inequality
%\demo: making sence of it all
%\me diarmids inequality
%\end{itemize}
\section{benntl's inequality}
Let
\begin{align}
X_1, \dots, X_n
\end{align}
be independent RVS with finite vaniance and $xi \leqslant b far b> 0$ almost surely for $i\leqslant n$ . let $v= \Sigma_{1=1}^{n} E xi^2.$ if,\\

S= $\Sigma_{1=1}^{N} xi- Exi$,then for any $\lambda > 0$\\
\begin{align}
log E[e^{\lambda s}\leqslant n. log(1+\frac{v}{nb^2} \phi (\lambda b))\leqslant \frac{v}{b^2} \phi (\lambda b) 
\end{align}
where 
\begin{align}
\phi (u)=e^u-u-1.,u\varepsilon R
\end{align}
\textbf{proof:} $\bullet$ let no assume b=1 $\rightarrow\\ 1$\\

$\bullet$ Observe that $u^2$ .$\phi(u)$ is a non decreacing function $\rightarrow 2$\\

from $1 or 2$ we can write \\
\begin{align}
(\lambda xi)^2. \phi(\lambda xi)\leqslant \lambda^2. \phi(\lambda)
\end{align}
	\begin{align}
\Rightarrow \phi(\lambda xi)\leqslant xi^2 \phi (\lambda)
\end{align}
\begin{align}
e^{\lambda xi}-\lambda xi-1 \leqslant xi^2 \phi (\lambda)
\end{align}
\begin{align}
E[e^{\lambda xi}-\lambda xi -1]\leqslant E[xi^2].\phi(\lambda).
\end{align}
\begin{align}
E[e^{\lambda xi}]\leqslant E[\lambda xi]+1+E[xi^2].\phi (\lambda)
\end{align}
\begin{align}
log E[e^{\lambda xi}]\leqslant log [E[\lambda xi]+1+E[xi^2] \phi (\lambda)]
\end{align}
\begin{align}
log E[e^{\lambda xi}- \lambda E[xi]\leqslant log(1+E[\lambda xi]+ E[xi^2].\phi (\lambda)
\end{align}
\begin{align}
\Sigma_{1=1}^{n} log [\frac{E[e^{\lambda xi}}{e^E{\lambda Ei}]\leqslant}] log (1+E[\lambda xi]+E[xi^2] \phi (\lambda) \lambda E[xi]]
\end{align}
\begin{align}
\psi_s(\lambda)\leqslant \Sigma_{1=1}^{n}
\end{align}
observe that log() is a concare function i.e.\\
\begin{align}
f(\lambda x+(1-\lambda). y) \geqslant \lambda. f(x)+(1-\lambda).f(y)-3
\end{align}
\begin{align}
\Sigma_{1=1}^{n}[1+\lambda .E[xi]+ E[xi^2].\phi (\lambda)]=
\end{align}
\begin{align}
\eta. \Sigma_{1=1}^{n} \frac{1}{n}.log [1+\lambda E[xi]+E[xi^2].\phi (\lambda)]
\end{align}
from-3
\begin{align}
\eta. \Sigma_{1=1}^{n} log[1+\lambda .E[xi]+E[xi^2]\phi (\lambda)]\leqslant
\end{align}
\begin{align}
\eta. log (\frac{1}{n}. \Sigma_{1=1}^{n}(1+\lambda E[xi]+E[xi]+E[xi^2].\phi (\lambda)))
\end{align}
\begin{align}
=\eta.log(1+\Sigma_{1=1}^{n} \lambda \frac{E[xi]}{n}+\frac{\vartheta}{n}\phi (\lambda)
\end{align}
Appling 4 and 2b give us\\
\begin{align}
\psi_s(\lambda)\leqslant \eta[log(1+[\Sigma_{1=1}^{n}\frac{\lambda.E[xi]}{n}+\frac{\vartheta}{n}.\phi (\lambda)) \frac{1}{n}\Sigma_{1=1}^{n} \lambda E[xi]]
\end{align}
observe log (1+x)$\leqslant$x $x \geqslant 0$.this helps us show\\ 
\begin{align}
\eta. [log (1+\Sigma_{1=1}^{n}\lambda \frac{E[xi]}{n}+\frac{\vartheta}{n}\phi(\lambda))-\frac{1}{n}\Sigma_{1=1}^{n}\lambda E[x]
\end{align}
\begin{align}
\leqslant n[\Sigma_{1=1}^{n}\frac{\lambda.E[xi}{n}+\frac{\vartheta}{n}.\phi (\lambda)-\frac{1}{n}\Sigma_{1=1} \lambda E[xi]]
\end{align}
\begin{align}
=\vartheta. \phi (\lambda)
\end{align}

\end{document}