\documentclass[journal,12pt,twocolumn]{IEEEtran}
%
\usepackage{setspace}
\usepackage{gensymb}
\usepackage{xcolor}
\usepackage{caption}
%\usepackage{subcaption}
%\doublespacing
\singlespacing

%\usepackage{graphicx}
%\usepackage{amssymb}
%\usepackage{relsize}
\usepackage[cmex10]{amsmath}
\usepackage{mathtools}
%\usepackage{amsthm}
%\interdisplaylinepenalty=2500
%\savesymbol{iint}
%\usepackage{txfonts}
%\restoresymbol{TXF}{iint}
%\usepackage{wasysym}
\usepackage{amsthm}
\usepackage{mathrsfs}
\usepackage{txfonts}
\usepackage{stfloats}
\usepackage{cite}
\usepackage{cases}
\usepackage{subfig}
%\usepackage{xtab}
\usepackage{longtable}
\usepackage{multirow}
%\usepackage{algorithm}
%\usepackage{algpseudocode}
\usepackage{enumitem}
\usepackage{mathtools}
%\usepackage{iithtlc}
%\usepackage[framemethod=tikz]{mdframed}
\usepackage{listings}


%\usepackage{stmaryrd}


%\usepackage{wasysym}
%\newcounter{MYtempeqncnt}
\DeclareMathOperator*{\Res}{Res}
%\renewcommand{\baselinestretch}{2}
\renewcommand\thesection{\arabic{section}}
\renewcommand\thesubsection{\thesection.\arabic{subsection}}
\renewcommand\thesubsubsection{\thesubsection.\arabic{subsubsection}}

\renewcommand\thesectiondis{\arabic{section}}
\renewcommand\thesubsectiondis{\thesectiondis.\arabic{subsection}}
\renewcommand\thesubsubsectiondis{\thesubsectiondis.\arabic{subsubsection}}

% correct bad hyphenation here
\hyphenation{op-tical net-works semi-conduc-tor}

\lstset{
language=Python,
frame=single, 
breaklines=true
}

%\lstset{
	%%basicstyle=\small\ttfamily\bfseries,
	%%numberstyle=\small\ttfamily,
	%language=python,
	%backgroundcolor=\color{white},
	%%frame=single,
	%%keywordstyle=\bfseries,
	%%breaklines=true,
	%%showstringspaces=false,
	%%xleftmargin=-10mm,
	%%aboveskip=-1mm,
	%%belowskip=0mm
%}

%\surroundwithmdframed[width=\columnwidth]{lstlisting
\begin{document}
%

\theoremstyle{definition}
\newtheorem{theorem}{Theorem}[section]
\newtheorem{problem}{Problem}[section]
\newtheorem{proposition}{Proposition}[section]
\newtheorem{lemma}{Lemma}[section]
\newtheorem{corollary}[theorem]{Corollary}
\newtheorem{example}{Example}[section]
\newtheorem{definition}{Definition}[section]
%\newtheorem{definition}{Definition}
%\newtheorem{algorithm}{Algorithm}[section]
%\newtheorem{cor}{Corollary}
\newcommand{\BEQA}{\begin{eqnarray}}
\newcommand{\EEQA}{\end{eqnarray}}
\newcommand{\define}{\stackrel{\triangle}{=}}

\bibliographystyle{IEEEtran}
%\bibliographystyle{ieeetr}

\providecommand{\nCr}[2]{\,^{#1}C_{#2}} % nCr
\providecommand{\nPr}[2]{\,^{#1}P_{#2}} % nPr
\providecommand{\mbf}{\mathbf}
\providecommand{\pr}[1]{\ensuremath{\Pr\left(#1\right)}}
\providecommand{\qfunc}[1]{\ensuremath{Q\left(#1\right)}}
\providecommand{\sbrak}[1]{\ensuremath{{}\left[#1\right]}}
\providecommand{\lsbrak}[1]{\ensuremath{{}\left[#1\right.}}
\providecommand{\rsbrak}[1]{\ensuremath{{}\left.#1\right]}}
\providecommand{\brak}[1]{\ensuremath{\left(#1\right)}}
\providecommand{\lbrak}[1]{\ensuremath{\left(#1\right.}}
\providecommand{\rbrak}[1]{\ensuremath{\left.#1\right)}}
\providecommand{\cbrak}[1]{\ensuremath{\left\{#1\right\}}}
\providecommand{\lcbrak}[1]{\ensuremath{\left\{#1\right.}}
\providecommand{\rcbrak}[1]{\ensuremath{\left.#1\right\}}}
\theoremstyle{remark}
\newtheorem{rem}{Remark}
\newcommand{\sgn}{\mathop{\mathrm{sgn}}}
\providecommand{\abs}[1]{\left\vert#1\right\vert}
\providecommand{\res}[1]{\Res\displaylimits_{#1}} 
\providecommand{\norm}[1]{\lVert#1\rVert}
\providecommand{\mtx}[1]{\mathbf{#1}}
\providecommand{\mean}[1]{E\left[ #1 \right]}
\providecommand{\fourier}{\overset{\mathcal{F}}{ \rightleftharpoons}}
%\providecommand{\hilbert}{\overset{\mathcal{H}}{ \rightleftharpoons}}
\providecommand{\system}{\overset{\mathcal{H}}{ \longleftrightarrow}}
	%\newcommand{\solution}[2]{\textbf{Solution:}{#1}}
\newcommand{\solution}{\noindent \textbf{Solution: }}
\providecommand{\dec}[2]{\ensuremath{\overset{#1}{\underset{#2}{\gtrless}}}}
\numberwithin{equation}{section}
%\numberwithin{equation}{problem}
%\numberwithin{problem}{subsection}
%\numberwithin{definition}{subsection}
\makeatletter
\@addtoreset{figure}{problem}
\makeatother

\let\StandardTheFigure\thefigure
%\renewcommand{\thefigure}{\theproblem.\arabic{figure}}
\renewcommand{\thefigure}{\theproblem}


%\numberwithin{figure}{subsection}

\def\putbox#1#2#3{\makebox[0in][l]{\makebox[#1][l]{}\raisebox{\baselineskip}[0in][0in]{\raisebox{#2}[0in][0in]{#3}}}}
     \def\rightbox#1{\makebox[0in][r]{#1}}
     \def\centbox#1{\makebox[0in]{#1}}
     \def\topbox#1{\raisebox{-\baselineskip}[0in][0in]{#1}}
     \def\midbox#1{\raisebox{-0.5\baselineskip}[0in][0in]{#1}}

\vspace{3cm}

\title{ 
%\logo{
EE5603:Conceention Inequalities
%}
%	\logo{python for Math Computing }
}
%\title{
%	\logo{Matrix Analysis through python}{\begin{center}\includegraphics[scale=.24]{tlc}\end{center}}{}{HAMDSP}
%}


% paper title
% can use linebreaks \\ within to get better formatting as desired
%\title{Matrix Analysis through python}
%
%
% author names and IEEE memberships
% note positions of commas and nonbreaking spaces ( ~ ) LaTeX will not break
% a structure at a ~ so this keeps an author's name from being broken across
% two lines.
% use \thanks{} to gain access to the first footnote area
% a separate \thanks must be used for each paragraph as LaTeX2e's \thanks
% was not built to handle multiple paragraphs
%

\author{J.~Balasubramaniam$^{\dagger}$ and G V V Sharma$^{*}$ %<-this  stops a space
\thanks{$\dagger$ The author is with the Department of Mathematics, IIT Hyderabad.  *The author is with the Department
of Electrical Engineering, IIT, Hyderabad
502285 India e-mail: \{jbala,gadepall\}@iith.ac.in. All material in the manuscript is released under GNU GPL.  Free to use for all.}% <-this % stops a space
%\thanks{J. Doe and J. Doe are with Anonymous University.}% <-this % stops a space
%\thanks{Manuscript received April 19, 2005; revised January 11, 2007.}}
}
% note the % following the last \IEEEmembership and also \thanks - 
% these prevent an unwanted space from occurring between the last author name
% and the end of the author line. i.e., if you had this:
% 
% \author{....lastname \thanks{...} \thanks{...} }
%                     ^------------^------------^----Do not want these spaces!
%
% a space would be appended to the last name and could cause every name on that
% line to be shifted left slightly. This is one of those "LaTeX things". For
% instance, "\textbf{A} \textbf{B}" will typeset as "A B" not "AB". To get
% "AB" then you have to do: "\textbf{A}\textbf{B}"
% \thanks is no different in this regard, so shield the last } of each \thanks
% that ends a line with a % and do not let a space in before the next \thanks.
% Spaces after \IEEEmembership other than the last one are OK (and needed) as
% you are supposed to have spaces between the names. For what it is worth,
% this is a minor point as most people would not even notice if the said evil
% space somehow managed to creep in.



% The paper headers
%\markboth{Journal of \LaTeX\ Class Files,~Vol.~6, No.~1, January~2007}%
%{Shell \MakeLowercase{\textit{et al.}}: Bare Demo of IEEEtran.cls for Journals}
% The only time the second header will appear is for the odd numbered pages
% after the title page when using the twoside option.
% 
% *** Note that you probably will NOT want to include the author's ***
% *** name in the headers of peer review papers.                   ***
% You can use \ifCLASSOPTIONpeerreview for conditional compilation here if
% you desire.




% If you want to put a publisher's ID mark on the page you can do it like
% this:
%\IEEEpubid{0000--0000/00\$00.00~\copyright~2007 IEEE}
% Remember, if you use this you must call \IEEEpubidadjcol in the second
% column for its text to clear the IEEEpubid mark.



% make the title area
\maketitle

%\documentclass{article}
%\usepackage{amsmath}
%\begin{document}
%\centerline{\textbf{EE5603:Conceention Inequalities}}

\section{Convergence}
\subsection{Definitions}
\begin{itemize}
\item Today:
\item {review basic inqulities}\\
\item {law of large numbers (proof)}\\
\item {demo on LLN and CLT}\\
\item {hoeffohing's inqulity:}\\

\textbf{Review:} Markor inequality: for a non-negative RVX and for any$\varepsilon$ $>0$\\
\begin{align}
\Pr(x\geqslant \varepsilon)\leqslant \frac{E{X}}{\varepsilon}\\
\end{align}
Chebyshow inequality:for any RVX with bounded variance,\\
\begin{align}
\Pr\brak{|x-E[X]|\geqslant\varepsilon} \leqslant \frac{Var(x)}{\varepsilon^2}\\
\end{align}
Cheruff bouned :for any RVX with bounded variance,and for any $t$ $>0,$\\
\begin{align}
\Pr(e^{tx}\geqslant e^{t\varepsilon})\leqslant \frac{E{e^{tx}}}{e^{t\varepsilon}}\\
\end{align}
\section{Law of large numbers (LLN)}
\begin{align}
\Pr(lim_{n\longrightarrow\infty}| \frac{1}{n}s_n-m| \geqslant \varepsilon)=0\\
\end{align}
$s_n=\Sigma_{1=1}^{n} X_i$ are iid RVS with mean $\mu$  and bounded variance $\sigma^2$\\

{applying chebyshev inequality},\\
\begin{align}
\Pr\brak{|\frac{s_n}{n}-\mu|\leqslant\varepsilon}\leqslant \frac{Var(\frac{s_n}{n})}{\varepsilon^2}\\
\end{align}
$Var(s_n)=\Sigma_{1=1}^{n} Var(x_i)=\frac{\sigma^2}{n}$\\
\begin{align}
\therefore \Pr(|\frac{s_n}{n}-\mu|\geqslant \varepsilon)\leqslant \frac{\sigma^2}{n\varepsilon^2}\\
\end{align}
$\Rightarrow lim_n \rightarrow \infty \Pr(|\frac{s_n}{n}-\mu|\geqslant \varepsilon) \leqslant lim_n\rightarrow\infty \frac{\sigma^2}{n\varepsilon^2}$\\
=0\\

\section{Hoeffding's inequalitity:}
\begin{align}
mativation:s_n=\Sigma_{1=1}^{n} X_i [x_i are iid]\\
\end{align}
\textbf{applying chebysher inequality} to  $\frac{s_n}{n}$ gives us.\\
\begin{align}
\Pr(|\frac{s_n}{n}-\mu| \geqslant\varepsilon)\leqslant \frac{\sigma^2}{n\varepsilon^2}\\
\end{align}
where $Var(X_i)=\sigma^2 E{X_i}=\mu$.\\
\textbf{ \item how about the tail problitity} $\Pr(|\frac{s_n}{n}-\mu \geqslant \varepsilon)?$\\

we know $Pr(\frac{s_n}{n}-\mu\geqslant \varepsilon)= \Pr(e^{\lambda(\frac{s_n}{n})}-\mu \geqslant e^{\lambda\varepsilon}$\\
apply
\begin{align}
\mu.I. \leqslant \frac{E[e^\lambda(\frac{s_n}{n}-\mu)]}{e^\lambda\varepsilon}\\
\end{align}
\begin{align}
=e^{-\lambda \varepsilon} E[e^\lambda[\frac{1}{n} \Sigma_{1=1}^{n} X_i-E[Xi]\\
\end{align}
\begin{align}
=e^-{\lambda\varepsilon} \Pi_{1=1}^{n} E[e^\lambda \frac{X_i-E[xi]}{n}]\\ 
\end{align}
Question: How canwe come with a toght bound for the \\
moment guerating function of (X-E[X]).\\
\section{hoeffding's lemma:}if x is a radom vaniable with E[X]=0 and a$\leqslant X \leqslant b$, then for a any s$>$0\\

$E[e^{sx}]\leqslant e\frac{s^2(b-a)^2}{8}$\\

  

\end{itemize}

\end{document}